\documentclass[a4paper,10pt]{article}
\usepackage[utf8]{inputenc}
\usepackage[parfill]{parskip}
\usepackage[left=1in,right=1in,top=0.65in,bottom=1in]{geometry}
\usepackage{setspace}
\usepackage{amsmath}
\usepackage{amssymb}
\usepackage{braket}
 
\newcommand{\cVec}[2]{(\begin{smallmatrix}#1\\#2\end{smallmatrix})}
 
\title{Research Proposal}
\author{Andreas Papageorgiou}
\date{}

\begin{document}
\maketitle
\thispagestyle{empty}

\onehalfspacing

Quantum computation has been studied mainly through the use of linear algebra. In this framework, a qubit can be expressed
using the Dirac notation as $\Ket{\psi} = \alpha \Ket{0} + \beta \Ket{1}$, where $\Ket{0} = \cVec{1}{0}$ and $\Ket{0} = \cVec{0}{1}$
form a basis of $\mathbb{C}^2$. An $n-$qubit system however, is defined and transformed in $\mathbb{C}^{2^n}$.
This exponential explosion is what restricts us from simulating such systems.

A qubit however can also be seen as a probability distribution, with the two basis vector
being the possible outcomes. This restricts $\alpha$ and $\beta$ such that $|\alpha|^2 + |\beta|^2 = 1$. This restriction
allows us to rewrite an arbitrary qubit as $\Ket{\psi} = \cos(\frac{\theta}{2})\Ket{0} + e^{i\phi}\sin(\frac{\theta}{2})\Ket{1}$,
which places the qubit as a point on a Bloch Sphere. From this, we can see that the two angles, $\theta$ and $\phi$ are sufficient
to represent the information held by a qubit.

We would like to examine a quantum system from the perspective of these angles. In other words, this can be seen as moving from
cartesian to polar coordinates. The major benefit of this is that it would allow us to exploit symmetries, in order to reduce
the amount of information needed to represent $n-$qubit systems, and therefore the amount of operations needed to transorm it.

By examining how arbitray qubits are joint, we can define a multidimensional bloch sphere, we can introduce transormations
and constraints between the angles that live in it, in order to efficiently express quantum operations.

To do this we would have to examine quantum operations, with respect to their effects on the angles of the qubits used.
We would be interested in a number of such operations.

This can be done by defining how a collection of qubits translates to a collection of angle-pairs, giving rise
to a multidimensional bloch sphere.


 \item 
 \item How measurement (i.e. projection) is performed, and how it affects the system.
\end{itemize} 


In other words, this can be seen as moving from
cartesian to polar coordinates. 



What is a qubit? A qubit can be expressed as a linear combination of two vectors $\binom{1}{0}$ and $\binom{0}{1}$ which form
one basis of a vector space in $\mathbb{C}$. In Dirac notation this can be expressed as $\alpha \Ket{0} + \beta \Ket{1}$.
Since we can also interprete a qubit as probability distribution, we know that $|\alpha|^2 + |\beta|^2 = 1$. This
constraint allows us to rewrite an arbitrary state of a qubit as $\Ket{\psi} = \cos(\frac{\theta}{2}) \Ket{0} + e^{i\phi}\sin(\frac{\theta}{2})\Ket{1}$,
which place a qubit as point on a Bloch Sphere. The two angles, $\theta$ and $\phi$, define all the information that
is needed in order to express a qubit. The aim of this project would be to examine these angles and how
quantum operations affect them. To do this we would have to look at how two and more bloch spheres interact with each other,
and define the space and rules that govern multidimensional bloch spheres and points that live on them.

To provide some motivation behind this, consider the case of two qubits with angles $\theta_1 = \theta_2 = 0$ and $\phi_1 = 0, \phi_2 = \frac{\pi}{2}$.
These qubits are clearly different since change in $\theta s$ would result in different points on the bloch sphere.
However using cartesian coordinates, the original qubits are both translated to $\binom{1}{0}$, which seems to be
missing a sense of direction, provided by the angle $\phi$. 

\end{document}