\documentclass[a4paper,10pt]{article}
\usepackage[utf8]{inputenc}
\usepackage[parfill]{parskip}
\usepackage[left=1in,right=1in,top=0.65in,bottom=1in]{geometry}
\usepackage{setspace}
\usepackage{amsmath}
\usepackage{amssymb}
\usepackage{braket}
 
\newcommand{\cVec}[2]{(\begin{smallmatrix}#1\\#2\end{smallmatrix})}
 
\title{Title}
\author{Authors}
\date{}

\begin{document}
\maketitle
\thispagestyle{empty}

\onehalfspacing

Quantum computation has been studied mainly through the use of linear algebra. In this framework, a qubit can be expressed
using the Dirac notation as $\Ket{\psi} = \alpha \Ket{0} + \beta \Ket{1}$, where $\Ket{0} = \cVec{1}{0}$ and $\Ket{0} = \cVec{0}{1}$
form a basis of $\mathbb{C}^2$. An $n-$qubit system however, is defined and transformed in $\mathbb{C}^{2^n}$.
This exponential explosion is what restricts us from simulating such systems.

A qubit however can also be seen as a probability distribution, with the two basis vector
being the possible outcomes. This restricts $\alpha$ and $\beta$ such that $|\alpha|^2 + |\beta|^2 = 1$. This
allows us to rewrite an arbitrary qubit as $\Ket{\psi} = \cos(\frac{\theta}{2})\Ket{0} + e^{i\phi}\sin(\frac{\theta}{2})\Ket{1}$,
which places the qubit as a point on a Bloch Sphere. From this, we can see that the two angles, $\theta$ and $\phi$ are sufficient
to represent the information held by a qubit.

We would like to examine a quantum system from the perspective of these angles. In other words, this can be seen as moving from
cartesian to polar coordinates. The major benefit of this is that it would allow us to exploit symmetries, in order to reduce
the amount of information needed to represent $n-$qubit systems, and therefore the amount of operations needed to transorm it.

%To provide some motivation to this, consider the case of two qubits with angles $\theta_1 = \theta_2 = 0$ and $\phi_1 = 0, \phi_2 = \frac{\pi}{2}$.
%These qubits are clearly different since change in $\theta s$ would result in different points on the bloch sphere.
%However using cartesian coordinates, the original qubits are both translated to $\binom{1}{0}$, which seems to be
%missing a sense of direction, provided by the angle $\phi$. 

\end{document}