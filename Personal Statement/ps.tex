\documentclass[a4paper,10pt]{article}
\usepackage[utf8]{inputenc}
\usepackage[parfill]{parskip}
\usepackage[left=1in,right=1in,top=0.65in,bottom=1in]{geometry}
\usepackage{setspace}


\title{Personal Statement}
\author{Andreas Papageorgiou}
\date{}

\begin{document}
\maketitle
\thispagestyle{empty}

\onehalfspacing

While growing up, I remember my self as always being curious. Me and my father, a teacher of physics, used to engage in coversations around
what things are, what they are made of, and how they work. He taught me that we can obtain answers of such
questions by making observations, and that the interpretations we give is how we define reality and what we know about it.
Most importanlty however, I learned that there are a lot of things that we do not yet know. This triggered my curiosity to find out.

During my early education, I loved subjects such as Physics and Mathematics. They provided me with the necessary tools
to solve problems in a very structured manner, and formed my way of thinking. Later on, I fell in love with programming.
The art of creating something from scratch, the complexity of problems encountered, as well as a personal urge to solve them
is what drove me to study Computing.

When I first heard about Quantum Computing, physics subjects such as space-time, the fourth dimension and multiple realities, were 
something fun for me to think about. However, the idea that we can harness and exploit these powers really intrigued me.
I started studying on my own, trying to understand the basic reasoning used and why it is better, aiming to provide my own
interpretations to things. The friction I had, helped me develop a vocabualry on the subject, understand the mathematics used,
and gain a fair amount of intuition around Quantum Computing. Although I haven't yet managed to fully understand it,
my passion has only grown bigger. This is why I decided to study this subject during my final year project, and also pursue a PhD in this area.

My dream is to find an efficient way to classically simulate the mysterious quantum mechanical properties of nature, or at least understand
why that is impossible. However, I believe that the only way for this to happen is to further examine and question what we know so far.
I would like to explore and also invent interesting ways with which we would use a quantum computer,
hoping that during the process my questions will be answered.

Having the opportunity to continue my studies  at Imperial College would be a great pleasure for me. Over the past four years, it has become
a second home to me and I wouldn't want to leave it. Not only that, but to share my ideas, influence and get influenced by great minds would be a thrilling
experience to me.

Finally, my goal is to not only learn about Quantum Computing, but also find ways to teach it. Wheather it would be through my work, through lecturing
or through software applications, I would like to find methods that would allow me to convey this knowledge in a simple and elegant manner.
The primary reason to this, is that looking for ways to teach a subject can help the teacher better understand it. Moreover, I believe there is a certain
pleasure into transfering knowledge, and watch as it developes in the minds of others.

Richard Feynman once said knowing something and knowing its name are two different things.
When it comes to Quantum Computing, I don't want to know just its name. I want to learn how it works, why it works and how else it could work.
With multiple discoveries on ways we can manipulate elementary particles, I believe our world is not far from seeing a universal quantum computer being realised.
I want to be there. I want to be ready to use it.

\end{document}